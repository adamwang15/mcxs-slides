\documentclass[notes,blackandwhite,mathsans]{beamer}

\usepackage{amsmath}
\usepackage{amssymb}
\usepackage{graphicx}
\usepackage{fancybox}
\usepackage{booktabs}
\usepackage{multirow,pxfonts}
\usepackage{array}
\usepackage{wrapfig}
\usepackage{changepage}
\usepackage{cmbright}
\usepackage{xcolor}
\usepackage{color}
\usepackage{enumitem}
\usepackage{ragged2e}
\usepackage{listings}

%\usetheme{Singapore} 	% Huge space, no names no nothing, no bars

% The default font theme installs a sans serif for all text of the presentation
%\usefonttheme{default}
%\setbeamercovered{invisible}
\beamertemplatenavigationsymbolsempty

\usepackage[T1]{fontenc}
\fontencoding{T1}  
\usepackage[utf8]{inputenc}


\lstset{basicstyle=\ttfamily\scriptsize,
         keywordstyle=\color{white}\ttfamily,
         commentstyle=\color{gray}\ttfamily,
         morecomment=[l][\color{gray}]{\#},
	emphstyle=\color{purple},
	emph={vec,mat}
}




\begin{document}
\fontfamily{pag}\selectfont
\setbeamerfont{title}{family=\fontfamily{pag}\selectfont}
\setbeamerfont{frametitle}{family=\fontfamily{pag}\selectfont}
\setbeamerfont{framesubtitle}{family=\fontfamily{pag}\selectfont}













{\setbeamercolor{background canvas}{bg=purple}
\begin{frame}

{\Large\color{black}Sharing early experiences of}\\ [0.5ex]
{\Large\color{black}programming with Rcpp}\\ [27ex]

Tomasz Wo\'zniak {\color{lightgray}University of Melbourne}\\[1ex]

{Macroeconometrics} {\color{lightgray}}\\

\end{frame}
}




{\setbeamercolor{background canvas}{bg=lightgray}
\begin{frame}{Rcpp package}

\bigskip \textbf{Rcpp} package facilitates the application of cpp for computations in R.

\bigskip {\color{darkgray}It provides interface for communication between R and cpp simplifying it.}

\bigskip {\color{darkgray}It is much easier to benefit from the best of the two worlds:}
\begin{description}
\smallskip\item[cpp] {\color{darkgray}code is compiled facilitating quick computations:}\\ {\color{purple}perfect for writing functions}

\smallskip\item[R] {\color{darkgray}code is interpreted and dynamic:}\\ {\color{purple}perfect for data analysis using functions written in cpp}
\end{description}

\end{frame}




\begin{frame}{Rcpp package}

In this presentation we are focusing on applications to Bayesian estimation that relies on:\small

\bigskip\textbf{elements of programming:} {\color{darkgray}functions, loops, etc.}

\smallskip\textbf{compatible object types:} {\color{darkgray}scalars, vectors, matrices, lists, etc.}

\smallskip\textbf{linear algebra:} {\color{darkgray}using library} Armadillo

\smallskip\textbf{random number generators:} {\color{darkgray}fast and reproducible using} Rcpp

\vspace{1cm}
\textbf{Rcpp} {\color{darkgray}is an R package providing the interface, object type compatibility, ..., and} {\color{purple}vectorised random number generators }

\smallskip\textbf{Armadillo} {\color{darkgray}is a cpp library for linear algebra with fantastic \href{http://arma.sourceforge.net/docs.html}{documentation online}} {\color{purple}}

\smallskip\textbf{RcppArmadillo} {\color{darkgray}is an R package providing simplified interface with} Armadillo {\color{darkgray}and providing compatibility with its object types}

\end{frame}








\begin{frame}{Rcpp package: learning resources}\small

\bigskip\textbf{book:} {\color{darkgray}Eddelbuettel (2013) Seamless R and C++ Integration with Rcpp}

\bigskip\textbf{bookdown:} {\color{darkgray}Tsuda (2020) \href{https://teuder.github.io/rcpp4everyone_en/}{Rcpp for everyone}}

\bigskip\textbf{vignettes:} {\color{darkgray}for packages} Rcpp {\color{darkgray}and} RcppArmadillo\\ {\color{darkgray}(published in JSS, CSDA, and TAS)}

\bigskip\textbf{datacamp course:} \href{https://learn.datacamp.com/courses/optimizing-r-code-with-rcpp}{\color{darkgray}Optimizing R Code with Rcpp} by Romain Fran\c{c}ois

\bigskip\textbf{online resources:} {\color{darkgray}\href{https://gallery.rcpp.org/}{RcppGallery}, \href{http://arma.sourceforge.net/docs.html}{Armadillo library documentation}, \href{http://stackoverflow.com}{stackoverflow.com}}

\end{frame}









{\setbeamercolor{background canvas}{bg=purple}
\begin{frame}

\begin{adjustwidth}{-0.5cm}{0cm}
\FlushLeft
\vspace{8.0cm}\LARGE
{{\color{black}A simple} {\color{lightgray}example}}
\end{adjustwidth}

\end{frame}
}



\begin{frame}[fragile]
\frametitle{Developing a function in a \texttt{.cpp} file}

\texttt{nicetry.cpp}\\
\lstinputlisting[language=C++, frame=trBL ]{nicetry_f.cpp} 

\end{frame}




\begin{frame}[fragile]
\frametitle{Using a function from a \texttt{.cpp} file}

\texttt{nicetry.cpp}\\
\lstinputlisting[language=C++, frame=trBL]{nicetry.cpp} % get rid of the 


\texttt{linreg.R}\\
\begin{lstlisting}[frame=trBL ]
library(Rcpp)
sourceCpp("nicetry.cpp")

x          = cbind(rep(1,5),1:5)
y          = x %*% c(1,2) + rnorm(5)
beta.hat   = nicetry(y, x)
\end{lstlisting}


\end{frame}











{\setbeamercolor{background canvas}{bg=purple}
\begin{frame}

\begin{adjustwidth}{-0.5cm}{0cm}
\FlushLeft
\vspace{8.0cm}\LARGE
{{\color{black}Simulation} {\color{lightgray}Smoother}}
\end{adjustwidth}

\end{frame}
}






\begin{frame}{Simulation smoother}

{\color{darkgray}Sample random draws from the multivariate normal distribution:}
$$ \mathcal{N}_T\left(\Omega^{-1}\alpha, \Omega^{-1}\right) $$

\begin{description}
\item[$\Omega$] - $T\times T$ {\color{darkgray}precision matrix that is band- or tri-diagonal}
\item[$\alpha$] - $T\times1$ {\color{darkgray}location vector}
\end{description}

\bigskip
{\color{darkgray}A random draw is computed by}
$$
L^{-1\prime}\left(L^{-1}\alpha + \epsilon\right)
$$
\begin{description}
\item[$L=\text{chol}(\Omega)$] {\color{darkgray}- is lower-triangular}
\item[$\epsilon$] - $T${\color{darkgray}-vector with independent standard normal draws}
\end{description}
\end{frame}









\begin{frame}[fragile]
\frametitle{R: simulation smoother \#1}
\texttt{rmvnorm\_solve.R}\\
\lstinputlisting[ frame=trBL]{rmvnorm_solve.R}
\end{frame}





\begin{frame}[fragile]
\frametitle{R: simulation smoother \#2}
\texttt{rmvnorm\_bandchol.R}\\
\lstinputlisting[frame=trBL]{rmvnorm_bandchol.R}
\end{frame}



\begin{frame}[fragile]
\frametitle{R: simulation smoother \#3}
\texttt{rmvnorm\_trichol.R}\\
\lstinputlisting[frame=trBL]{rmvnorm_trichol.R}
\end{frame}





\begin{frame}[fragile]
%\frametitle{Rcpp: simulation smoother \#1}
\texttt{rmvnorm\_arma\_inv.cpp}\\
\lstinputlisting[language=C++, linerange={1-24}, frame=trBL, linewidth=320pt]{rmvnorm_arma_inv.cpp}
\end{frame}



\begin{frame}[fragile]
%\frametitle{Rcpp: simulation smoother \#2}
\texttt{rmvnorm\_arma\_solve.cpp}\\
\lstinputlisting[language=C++, linerange={1-25}, frame=trBL]{rmvnorm_arma_solve.cpp}
\end{frame}






\begin{frame}[fragile]
%\frametitle{Rcpp: simulation smoother \#3}
\begin{adjustwidth}{-0.6cm}{0cm}
\texttt{rmvnorm\_arma\_stochvol.cpp}\\ 
\lstinputlisting[language=C++, linerange={56-78}]{rmvnorm_arma_stochvol.cpp}
\end{adjustwidth}
\end{frame}






\begin{frame}[fragile]
\frametitle{Simulation smoother}
\texttt{SimulationSmoother.R}\\ 
\lstinputlisting[ linerange={3-21}, frame=trBL, linewidth=320pt]{SimulationSmoother.R}
\end{frame}


\begin{frame}[fragile]
\frametitle{Simulation smoother}
\texttt{SimulationSmoother.R}\\ 
\lstinputlisting[linerange={23-32}, frame=trBL]{SimulationSmoother.R}

\bigskip
\begin{adjustwidth}{-0.4cm}{0cm}
\scriptsize
\begin{verbatim}
Unit: milliseconds
      expr       min        lq      mean    median        uq        max neval
   R.solve 15.012696 16.022046 19.049814 16.741754 21.296232  41.665376   100
    R.band  5.521327  6.298791 13.280026  6.719498  9.619501 117.470313   100
 R.tridiag  3.674740  4.004751  8.079238  4.449233  4.916731 178.783048   100
   cpp.inv 10.622570 11.113882 12.171498 11.594380 12.052189  21.046931   100
   cpp.sol  2.024739  2.183631  2.789788  2.808968  2.971318   6.256791   100
   cpp.sto  1.556249  1.621727  1.875903  1.882130  1.988038   3.034821   100
\end{verbatim}
\end{adjustwidth}
\end{frame}






}











{\setbeamercolor{background canvas}{bg=purple}
\begin{frame}

\begin{adjustwidth}{-0.5cm}{0cm}
\FlushLeft
\vspace{8.0cm}\LARGE
{{\color{black}Gibbs sampler} {\color{lightgray}for a local-level model}}
\end{adjustwidth}

\end{frame}
}




{\setbeamercolor{background canvas}{bg=lightgray}
\begin{frame}{Gibbs sampler for a local-level model}

The model for a unit-root non-stationary variable:
\begin{align*}
y_t &= \mu_t + \epsilon_t & \epsilon_t &\sim\mathcal{N}\left(0,\sigma^2\right)\\
\mu_t &= \mu_{t-1} + m_t & m_t &\sim\mathcal{N}\left(0,\sigma_m^2\right)
\end{align*}

Gibbs sampler:\footnotesize
\begin{align*}
\mu|y, \mu_0, \sigma^2, \sigma_n^2 &\sim\mathcal{N}_T\left( \overline{V}^{-1}\overline{\mu}, \overline{V}^{-1} \right)\\[1ex]
\overline{V}^{-1} &= \sigma^{-2}I_T + \sigma_m^{-2}H'H\\
\overline{\mu} &= \sigma^{-2}y + \sigma_m^{-2}\mu_0 e_{1}\\[1ex]
\mu_0|y, \mu_1, \sigma_m,^2 &\sim\mathcal{N}\left( \left( \sigma_m^{-2} + \underline{v}^{-1} \right)^{-1}, \left( \sigma_m^{-2} + \underline{v}^{-1} \right)^{-1}\sigma_m^{-2}\mu_1 \right)\\
\sigma^2|y,\mu
 &\sim\mathcal{IG}2\left(\underline{s} + (y-\mu)'(y-\mu), \underline{\nu} + T  \right)\\
\sigma_m^2|\mu, \mu_0 &\sim\mathcal{IG}2\left(\underline{s} + (H\mu-\mu_0e_{1})'(H\mu-\mu_0e_{1}), \underline{\nu} + T  \right)
\end{align*}

\end{frame}





\begin{frame}[fragile]
%\frametitle{Gibbs sampler for a local-level model}
\texttt{LL\_arma\_solve.cpp - part 1}\\ 
\lstinputlisting[language=C++, linerange={1-21}, frame=trBL, linewidth=320pt]{LL_arma_solve.cpp}
\end{frame}



\begin{frame}[fragile]
%\frametitle{Gibbs sampler for a local-level model}
\texttt{LL\_arma\_solve.cpp - part 2}\\ 
\lstinputlisting[language=C++, linerange={22-38}, frame=trBL, linewidth=320pt]{LL_arma_solve.cpp}
\end{frame}



\begin{frame}[fragile]
%\frametitle{Gibbs sampler for a local-level model}
\texttt{LL\_arma\_solve.cpp - part 3}\\ 
\begin{adjustwidth}{-0.4cm}{0cm}
\lstinputlisting[language=C++, linerange={40-62}, frame=trBL, linewidth=335pt]{LL_arma_solve.cpp}
\end{adjustwidth}
\end{frame}



\begin{frame}[fragile]
%\frametitle{Gibbs sampler for a local-level model}
\texttt{LL\_arma\_solve.cpp - part 4}\\ 
\begin{adjustwidth}{-0.4cm}{0cm}
\lstinputlisting[language=C++, linerange={64-81}, frame=trBL, linewidth=335pt]{LL_arma_solve.cpp}
\end{adjustwidth}
\end{frame}





\begin{frame}[fragile]
\frametitle{Gibbs sampler for a local-level model}
\texttt{LL.R}\\ 
\lstinputlisting[linerange={27-34}, frame=trBL]{LL.R}

\bigskip
\begin{adjustwidth}{-0.4cm}{0cm}
\scriptsize
\begin{verbatim}
Unit: milliseconds
    expr        min         lq       mean     median         uq       max neval
   R.not 127.266986 132.328077 155.005314 138.435266 154.022634 327.33741   100
   R.ban  46.167909  49.679159  95.663136  55.453045 159.034735 324.58272   100
   R.tri  19.089019  22.110003  41.142920  26.470268  29.056582 150.71913   100
 cpp.inv  10.580124  11.180750  12.486657  11.937587  12.845465  26.01354   100
 cpp.sol  10.542156  11.145461  12.291518  11.682092  12.380952  27.42871   100
 cpp.sto   8.177979   8.542281   9.554215   9.115295   9.739227  19.19876   100
\end{verbatim}
\end{adjustwidth}

\end{frame}













{\setbeamercolor{background canvas}{bg=purple}
\begin{frame}{What's next?}

\vspace{0.5cm}
\begin{center}
\LARGE{\color{black}Rewrite all your code in Rcpp!}
\end{center}
\end{frame}
}




\end{document} 