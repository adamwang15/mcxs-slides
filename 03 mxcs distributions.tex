\documentclass[10pt]{article}
\usepackage{amsmath}
\usepackage{amssymb}
\usepackage{amsfonts}
\usepackage{natbib}
\usepackage{nopageno}
\usepackage{hyperref}
\hypersetup{linkcolor=blue,citecolor=blue,filecolor=black,urlcolor=blue} 
\usepackage{graphicx}
\usepackage{fancybox}
\usepackage{booktabs}
\usepackage{multicol}
\usepackage{color}
\usepackage[top=1cm, bottom=2cm, left=1cm, right=1cm]{geometry}
\usepackage[none]{hyphenat}
%\usepackage{changepage}
\usepackage{fancyhdr}
\usepackage{cmbright}
\usepackage{enumitem}


\pagestyle{fancy}
\fancyhf{}
\renewcommand{\headrulewidth}{0pt}
\renewcommand{\footrulewidth}{0pt}
%\newcommand{\helv}{\fontfamily{pag}\fontsize{10}{11}\selectfont}
%\rhead{Share\LaTeX}
%\lhead{Guides and tutorials}
\lfoot{\rule{5cm}{.1pt}\\  Tomasz Wo\'zniak $\bullet$ e-mail: \href{mailto:tomasz.wozniak@unimelb.edu.au}{tomasz.wozniak@unimelb.edu.au} $\bullet$ website: \href{https://github.com/donotdespair}{github.com/donotdespair}\hspace{4cm} \thepage}



\begin{document}


%\begin{center}
%\large
%Macroeconometrics - ECOM90007, ECOM40003 \\
%\Large \textbf{Subject Guide - Semester 1, 2019} \\
%{\large by Tomasz Wo\'zniak}\\ 
%\end{center}


\bigskip\noindent\textbf{\LARGE Macroeconometrics: ECOM90007}

\smallskip\noindent by Tomasz Wo\'zniak $\bullet$ Department of Economics $\bullet$  University of Melbourne

\smallskip\noindent Semester 1, 2023

\smallskip\noindent\rule{5cm}{.1pt}

\normalsize
\bigskip\noindent\textbf{\Large Notes from the whiteboard.} 

\smallskip\noindent These are the whiteboard notes from lecture 3.

\bigskip\noindent\textbf{\large Definitions.} 

\smallskip\noindent The transformations below can be understood as definitions, decompositions, and the way to construct the distributions.


\smallskip\noindent Consider random variables $X$ and $Y$. 
\begin{align*}
p(X) &\text{ -- denotes a marginal distribution of } X \\
p(X,Y) &\text{ -- denotes a joint distribution of } X \text{ and } Y\\
p(X\mid Y) &\text{ -- denotes a conditional distribution of } X \text{ given } Y
\end{align*}


\smallskip\noindent The joint distribution can be decomposed as or be constructed from a conditional and marginal distribution
\begin{align*}
p(X,Y) & = p(X\mid Y) p(Y)\\
&= p(Y\mid X)p(X)
\end{align*}

\smallskip\noindent If $X$ and $Y$ are independent then their joint distribution can be constructed as a product of marginal distributions:
\begin{align*}
p(X,Y) & = p(X) p(Y)
\end{align*}

\smallskip\noindent A conditional distribution is constructed as:
\begin{align*}
p(X\mid Y) & = \frac{p(X, Y)}{p(Y)}\\
p(Y\mid X) & = \frac{p(X, Y)}{p(X)}
\end{align*}

\smallskip\noindent Marginal distributions can be constructed as:
\begin{align*}
p(X) & = \frac{p(X, Y)}{p(Y\mid X)}\\
&= \int p(X,Y)dY\\
p(Y) & = \frac{p(X, Y)}{p(X\mid Y)}\\
&= \int p(X,Y)dX
\end{align*}

\bigskip\noindent\textbf{\large Expected values.} 

\smallskip\noindent 
\begin{align*}
E(X) & = \int Xp(X)dX\\
E(X\mid Y) & = \int Xp(X\mid Y)dX\\
E\left(X^2\right) & = \int X^2p(X)dX\\
E(f(X)) & = \int f(X)p(X)dX
\end{align*}

\smallskip\noindent\textbf{Task.} Using the formulae above write down how you would compute/construct the variance of $X$, $Var(X)$.

\end{document}






