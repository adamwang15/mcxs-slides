\documentclass[notes,blackandwhite,mathsans,usenames,dvipsnames]{beamer}

\usepackage{amsmath}
\usepackage{amssymb}
\usepackage{graphicx}
\usepackage{fancybox}
\usepackage{booktabs}
\usepackage{multirow,pxfonts}
\usepackage{cmbright}
\usepackage{xcolor}
\usepackage{color}
\usepackage{enumitem}
\usepackage{animate}
\usepackage{changepage}

\usepackage[T1]{fontenc}
\fontencoding{T1}  
\usepackage[utf8]{inputenc}


\usefonttheme{default}
\setbeamercovered{invisible}
\beamertemplatenavigationsymbolsempty

\makeatletter
\setbeamertemplate{footline}
{
  \leavevmode
  \hbox{
  \begin{beamercolorbox}[wd=0.97\paperwidth,ht=2.25ex,dp=2ex,right]{}
{\color{mcxs2} \insertframenumber{} / \inserttotalframenumber}
  \end{beamercolorbox}}%
}


\definecolor{mcxs1}{HTML}{05386B}
\definecolor{mcxs2}{HTML}{379683}
\definecolor{mcxs3}{HTML}{5CDB95}
\definecolor{mcxs4}{HTML}{8EE4AF}
\definecolor{mcxs5}{HTML}{EDF5E1}
\setbeamercolor{frametitle}{fg=mcxs2}
\AtBeginDocument{\color{mcxs1}}
\setbeamertemplate{itemize item}[triangle]

\begin{document}
%\fontfamily{pag}\selectfont
%\setbeamerfont{title}{family=\fontfamily{pag}\selectfont}
%\setbeamerfont{frametitle}{family=\fontfamily{pag}\selectfont}
%\setbeamerfont{framesubtitle}{family=\fontfamily{pag}\selectfont}







{\setbeamercolor{background canvas}{bg=purple}
\begin{frame}

\vspace{1cm}
\begin{tabular}{rl}
&\textbf{\LARGE\color{mcxs2} Macroeconometrics}\\[8ex]
\textbf{\Large\color{mcxs2}Lecture 16}&\textbf{\Large\color{mcxs1}Unobserved Component models}\\[19ex]
&\textbf{Tomasz Wo\'zniak}\\[1ex]
&{\small\color{mcxs1} Department of Economics}\\
&{\small\color{mcxs1}University of Melbourne}
\end{tabular}

\end{frame}
}




{\setbeamercolor{background canvas}{bg=purple}
\begin{frame}

\vspace{1cm}\textbf{\color{mcxs1}A simple Unobserved Component model}

\bigskip\textbf{\color{lightgray}Identification and unconditional moments}

\bigskip\textbf{\color{mcxs1}UC model and Beveridge-Nelson decomposition}


\vspace{1cm} Useful readings: \small

\smallskip{\color{lightgray}Morley, Nelson, Zivot (2003) Why Are the Beveridge-Nelson and Unobserved-Components Decompositions of GDP so Different?\\ Review of Economics and Statistics}

\end{frame}
}




{\setbeamercolor{background canvas}{bg=purple}
\begin{frame}

\bigskip\textbf{\color{mcxs1}Objectives.}
\begin{itemize}[label=$\blacktriangleright$]
\item {\color{mcxs1}To present the elements of the unobserved component model}
\item {\color{mcxs1}To provide framework for the business-cycle decomposition }
\item {\color{mcxs1}To look at the identification of the parameters of UC models}
\end{itemize}

\bigskip\textbf{\color{lightgray}Learning outcomes.}
\begin{itemize}[label=$\blacktriangleright$]
\item {\color{lightgray}Implementing decomposition into unit-root stationary and non-stationary component}
\item {\color{lightgray}Specifying the cycle component}
\item {\color{lightgray}Understanding stochastic trend through the Beveridge-Nelson decomposition}
\end{itemize}

\end{frame}
}
 


{\setbeamercolor{background canvas}{bg=purple}
\begin{frame}


\bigskip
\begin{center}
\begin{tabular}{ c l}
\toprule 
\multicolumn{2}{c}{\textbf{Modeling Trend Inflation}}\\
16  & Unobserved Component models \\
17  & Bayesian estimation using precision sampler \\
18  & Bayesian estimation using precision sampler \\
19  & Modeling trend inflation \\[1ex]
\multicolumn{2}{c}{\textbf{Modeling Conditional Heteroskedasticity}}\\
20  & Stochastic Volatility models \\
21  & Bayesian estimation using auxiliary mixtures \\[1ex]
\bottomrule
\end{tabular}
\end{center}


\end{frame}
}




{\setbeamercolor{background canvas}{bg=purple}
\begin{frame}

\begin{adjustwidth}{-0.5cm}{0cm}
\vspace{8.3cm}\Large
\textbf{{\color{lightgray}A simple} {\color{mcxs1}Unobserved Component} {\color{lightgray} model}}
\end{adjustwidth}

\end{frame}
}



\begin{frame}{A simple Unobserved Component model}
$$
\begin{array}{rlll}
y_t &= \tau_t + \epsilon_t&&\text{\color{mcxs2}-- measurement equation}\\[1ex]
\tau_t &= \mu + \tau_{t-1} + \eta_t&&\text{\color{mcxs2}-- state equation for }\tau_t\\[1ex]
\alpha_p(L)\epsilon_t &=  e_t&&\text{\color{mcxs2}-- state equation for }\epsilon_t\\[1ex]
\eta_t|Y_{t-1} &\sim iid\mathcal{N}\left(0,\sigma_\eta^2\right)&&\text{\color{mcxs2}-- conditional distribution}\\[1ex]
e_t |Y_{t-1} &\sim iid\mathcal{N}\left(0,\sigma_e^2\right)&&\text{\color{mcxs2}-- conditional distribution}\\[5ex]
\alpha_p(L) &= 1-\alpha_1L - \dots - \alpha_pL^p&&\text{\color{mcxs2}-- lag polynomial}\\[1ex]
\alpha_p(z)&=0\text{, } \forall |z|>1\text{ and }z\in\mathbb{C}&&\text{\color{mcxs2}-- stationarity restriction}\\[1ex]
\sigma_{\eta e}&= \mathbb{C}ov[\eta_t,e_t]&&\text{\color{mcxs2}-- potential correlation}\\[1ex]
\theta&=\left(\mu,\alpha_1,\dots,\alpha_p,\sigma_\eta^2,\sigma_e^2\right)&&\text{\color{mcxs2}-- parameters}\\
\end{array}
$$
\end{frame}





\begin{frame}{A simple Unobserved Component model}

\begin{description}
\item[$y_t$] {\color{mcxs2}-- an observation on a scalar random variable at time $t$ }
\item[$\tau_t$] {\color{mcxs2}-- trend component -- unit-root non-stationary}
\item[$\epsilon_t$] {\color{mcxs2}-- unit-root stationary component }
\item[$\eta_t, e_t$] {\color{mcxs2}-- Gaussian white noise error terms }
\item[$\mu$] {\color{mcxs2}-- a drift  parameter}
\item[$Y_{t-1}$] {\color{mcxs2}-- information set}
\end{description}
\end{frame}


\begin{frame}{A simple Unobserved Component model}

\textbf{Measurement equation.}
$$y_t = \tau_t + \epsilon_t$$

\begin{description}
\item[Decomposes] {\color{mcxs2}the process into a unit-root non-stationary and a~unit-root stationary component}
\item[Contains measurements] {\color{mcxs2}on variable} $y_t$ {\color{mcxs2}on the left-hand side of the equation}
\item[Includes latent processes] $\tau_t$ {\color{mcxs2}and} $\epsilon_t$ {\color{mcxs2}on the right-hand side}
\end{description}
\end{frame}


\begin{frame}{A simple Unobserved Component model}

\bigskip
\textbf{Trend component.}
\begin{align*}
\tau_t &= \mu + \tau_{t-1} + \eta_t\\
\eta_t|Y_{t-1} &\sim iid\mathcal{N}\left(0,\sigma_\eta^2\right)
\end{align*}

%\small
\bigskip\begin{description}
\item[Trend process] {\color{mcxs2}is specified as a Gaussian random walk with drift process }
\item[Captures] {\color{mcxs2}the long-run forecast of} $y_t$ {\color{mcxs2}at an infinite horizon with the forecast origin} $t$
\item[The drift] {\color{mcxs2}captures the deterministic time trend slope of} $y_t$ {\color{mcxs2}and is often specified to change over time to represent a~structural break in the deterministic trend or allow } $y_t$ {\color{mcxs2}to have two unit roots.  The extended specification is often given by} 
\begin{align*}
\tau_t &= \mu_t + \tau_{t-1} + \eta_t\\
\mu_t &= \mu_{t-1} + m_t\\
m_t|Y_{t-1} &\sim iid\mathcal{N}\left(0,\sigma_m^2\right)
\end{align*}
\end{description}
\end{frame}



\begin{frame}{A simple Unobserved Component model}

\textbf{Unit-root stationary component.}
\begin{align*}
\alpha_p(L)\epsilon_t &=  e_t\\[1ex]
e_t |Y_{t-1} &\sim iid\mathcal{N}\left(0,\sigma_e^2\right)\\[1ex]
\alpha_p(z)&=0,\qquad \forall |z|>1\text{{\color{mcxs2}, and} }z\in\mathbb{C}
\end{align*}

\begin{description}
\item[Captures] {\color{mcxs2}the deviations from the long-run trend}
\item[Exhibits persistence] {\color{mcxs2}which allows to smoothen the trend }
\item[UC model] {\color{mcxs2}is called a} {\color{purple}local level model} {\color{mcxs2}when} $p=0\text{\color{mcxs2}, or }p=1$
\item[$\epsilon_t$ is called a cycle] {\color{mcxs2}if } $p\geq2$ {\color{mcxs2}when it can capture cyclical patterns in} $\epsilon_t$
\end{description}
\end{frame}






\begin{frame}{A simple Unobserved Component model}

\bigskip
\textbf{Unobserved component model.}
$$
\begin{array}{rl}
y_t &= \tau_t + \epsilon_t\\[1ex]
\tau_t &= \mu + \tau_{t-1} + \eta_t\\[1ex]
\alpha_p(L)\epsilon_t &=  e_t\\[1ex]
\eta_t|Y_{t-1} &\sim iid\mathcal{N}\left(0,\sigma_\eta^2\right)\\[1ex]
e_t |Y_{t-1} &\sim iid\mathcal{N}\left(0,\sigma_e^2\right)
\end{array}
$$

\bigskip\textbf{Autoregressive model} obtained by imposing $\sigma_\eta^2=0$
$$
\begin{array}{rl}
y_t -\mu t &= \epsilon_t\\[1ex]
\alpha_p(L)\epsilon_t &=  e_t\\[1ex]
e_t |Y_{t-1} &\sim iid\mathcal{N}\left(0,\sigma_e^2\right)
\end{array}
$$


\end{frame}








{\setbeamercolor{background canvas}{bg=purple}
\begin{frame}

\begin{adjustwidth}{-0.5cm}{0cm}
\vspace{8.3cm}\Large
\textbf{{\color{lightgray}Identification and} {\color{mcxs1}unconditional moments}}
\end{adjustwidth}

\end{frame}
}





\begin{frame}{Unconditional moments}

\begin{description}
\item[Unobserved component model] {\color{mcxs2}captures a~single unit-root of} $y_t$
\item[Unconditional moments] {\color{mcxs2}for} $y_t$ {\color{mcxs2}depend on time and the variances can be shown to not be well-specified\\ -- they take infinite values}
\item[Consider a model] {\color{mcxs2}for the first differences} $\Delta y_t$
\begin{align*}
\Delta y_t &= \Delta \tau_t + \Delta \epsilon_t\\[1ex]
\Delta \tau_t &= \mu + \eta_t\\[1ex]
\alpha_p(L)\Delta \epsilon_t &= \Delta e_t\\[1ex]
\Delta e_t|Y_{t-1}&\sim\mathcal{N}\left(0,2\sigma_e^2\right)
\end{align*}
{\color{mcxs2}where} $\Delta=1-L$
\end{description}

\end{frame}


\begin{frame}{Unconditional moments}

\begin{description}
\item[Rewrite] {\color{mcxs2}the system as}
\begin{align*}
\alpha_p(L)\Delta \epsilon_t &=  \Delta e_t\\[1ex]
\Delta \epsilon_t &= \Delta y_t - \mu - \eta_t 
\end{align*}
\item[Plug in] {\color{mcxs2}the second equation into the first one and reorganize elements}
\begin{align*}
\alpha_p(L)(\Delta y_t - \mu) &= \alpha_p(L)\eta_t + e_t - e_{t-1}
\end{align*}
\item[To obtain] {\color{mcxs2}the} {\color{purple}ARIMA($p,1,\max\{p,1\}$) representation of the UC model}

\bigskip\item[ARIMA(p,d,q)] {\color{mcxs2}model is given by}
$$ \alpha_p(L)\Delta^d (y_t-\mu) = \beta_q(L)\varepsilon_t $$
{\color{mcxs2}where} $d$ {\color{mcxs2}is the integration order} $y_t\sim I(d)$ {\color{mcxs2}and} $\Delta^d = (1-L)^d$
\end{description}

\end{frame}




\begin{frame}{Unconditional moments}

\textbf{Unconditional mean.}
\begin{align*}
\alpha_p(L)(\Delta y_t-\mu) &= \alpha_p(L)\eta_t + \Delta e_t\\[2ex]
\alpha_p(L)\mathbb{E}[\Delta y_t-\mu] &= \alpha_p(L)\mathbb{E}[\eta_t] + \mathbb{E}[\Delta e_t]\\
&\overset{LIE}{=} 0\\[2ex]
\mathbb{E}[\Delta y_t] &=\mu
\end{align*}

\bigskip\begin{description}
\item[$\mu$] {\color{mcxs2}is the average growth rate of} $y_t$ {\color{mcxs2}over one period}
\end{description}

\end{frame}



\begin{frame}{Identification: matching autocovariances}

$$
\alpha_p(L)(\Delta y_t - \mu) = \alpha_p(L)\eta_t + \Delta e_t
$$

\begin{description}
\item[LHS] {\color{mcxs2}of the equation specifies a standard AR dynamics of} $\Delta y_t - \mu$
\item[LHS] {\color{mcxs2}parameters} $\alpha_1,\dots,\alpha_p$ {\color{mcxs2}can be easily estimated}
\item[RHS] {\color{mcxs2}of the equation specifies the MA($\max\{p,1\}$) component}
$\beta_{\max\{p,1\}}(L) = 1+\beta_1L + \dots + \beta_{\max\{p,1\}}L^{\max\{p,1\}}$
\item[RHS] {\color{mcxs2}of the equation is written as}
$$ rhs_t = \alpha_p(L)\eta_t + e_t - e_{t-1} $$
\item[Derive the relationship] {\color{mcxs2}between the parameters on the equation above to the unconditional autocovariances of} $rhs_t$ {\color{mcxs2}for two cases} $p=1$ {\color{mcxs2}and} $p=2$ {\color{mcxs2}to estimate} $\sigma_\eta^2,\sigma_e^2,\sigma_{\eta e}$
\end{description}

\end{frame}






\begin{frame}{Identification: matching autocovariances}

\textbf{Case 1:} $p=1$ \scriptsize
\begin{align*}
rhs_t &= \eta_t - \alpha_1 \eta_{t-1} + e_t - e_{t-1}\\[3ex]
\gamma_s^r &= \mathbb{E}[rhs_t \cdot rhs_{t-s}]\\[3ex]
s=0 \qquad \gamma_0^r &= \mathbb{E}\left[rhs_t ^2\right] = \mathbb{E}\left[(\eta_t - \alpha_1 \eta_{t-1} + e_t - e_{t-1})^2\right]\\
&= \left(1+\alpha_1^2\right)\sigma^2_\eta + 2\sigma^2_e + 2\left(1+\alpha_1\right)\sigma_{\eta e}\\[3ex]
s=1 \qquad \gamma_1^r &= \mathbb{E}\left[rhs_t \cdot rhs_{t-1}\right] \\
&= \mathbb{E}\left[(\eta_t - \alpha_1 \eta_{t-1} + e_t - e_{t-1})(\eta_{t-1} - \alpha_1 \eta_{t-2} + e_{t-1} - e_{t-2})\right]\\
&= -\alpha_1\sigma^2_\eta -\sigma^2_e -\left(1+\alpha_1\right)\sigma_{\eta e}\\[3ex]
s\geq 2 \qquad \gamma_s^r &=0
\end{align*}

\end{frame}



\begin{frame}{Identification: matching autocovariances}

\textbf{Case 1:} $p=1$ 
\begin{align*}
\begin{bmatrix} \gamma^r_0 \\ \gamma^r_1 \end{bmatrix} = 
\begin{bmatrix} 1+\alpha_1^2 & 2 & 2(1+\alpha_1) \\  -\alpha_1 & -1 & -(1+\alpha_1) \end{bmatrix}
\begin{bmatrix} \sigma^2_\eta \\ \sigma^2_e \\ \sigma_{\eta e} \end{bmatrix}
\end{align*}

\begin{description}
\item[Autocovariances] {\color{mcxs2}on the LHS are features of data and the autoregressive parameter} $\alpha_1$
\item[Variances and covariance] {\color{mcxs2}on the RHS are the parameters of the UC model}
\item[The UC model] {\color{mcxs2}is} {\color{purple}not identified} {\color{mcxs2}and one identifying restriction is required}
\item[Identifying restriction] {\color{mcxs2}is often chose to be} $\sigma_{\eta e}=0$ {\color{mcxs2}imposing the independence of the shocks of the model}
\end{description}

\end{frame}




\begin{frame}{Identification: matching autocovariances}

\bigskip\textbf{Case 2:} $p=2$ \scriptsize
\begin{align*}
rhs_t &= \eta_t - \alpha_1 \eta_{t-1} - \alpha_2 \eta_{t-2} + e_t - e_{t-1}\\[3ex]
\gamma_s^r &= \mathbb{E}[rhs_t \cdot rhs_{t-s}]\\[3ex]
\gamma_0^r &= \mathbb{E}\left[rhs_t ^2\right] = \mathbb{E}\left[(\eta_t - \alpha_1 \eta_{t-1}- \alpha_2 \eta_{t-2} + e_t - e_{t-1})^2\right]\\
&= \left(1+\alpha_1^2+\alpha_2^2\right)\sigma^2_\eta + 2\sigma^2_e + 2\left(1+\alpha_1\right)\sigma_{\eta e}\\[3ex]
\gamma_1^r &= \mathbb{E}\left[rhs_t \cdot rhs_{t-1}\right]\\
& = \mathbb{E}\left[(\eta_t - \alpha_1 \eta_{t-1}- \alpha_2 \eta_{t-2} + e_t - e_{t-1})(\eta_{t-1} - \alpha_1 \eta_{t-2}- \alpha_2 \eta_{t-3} + e_{t-1} - e_{t-2})\right]\\
&= \alpha_1\left(\alpha_2-1\right)\sigma^2_\eta -\sigma^2_e - \left(1+\alpha_1-\alpha_2\right)\sigma_{\eta e}\\[3ex]
\gamma_2^r &= \mathbb{E}\left[rhs_t \cdot rhs_{t-2}\right]\\
& = \mathbb{E}\left[(\eta_t - \alpha_1 \eta_{t-1}- \alpha_2 \eta_{t-2} + e_t - e_{t-1})(\eta_{t-2} - \alpha_1 \eta_{t-3}- \alpha_2 \eta_{t-4} + e_{t-2} - e_{t-3})\right]\\
&= -\alpha_2\sigma^2_\eta -\alpha_2\sigma_{\eta e}\\[3ex]
\gamma_s^r &= 0 \text{ for } s\geq3
\end{align*}

\end{frame}


\begin{frame}{Identification: matching autocovariances}

\textbf{Case 2:} $p=2$ 
\begin{align*}
\begin{bmatrix} \gamma^r_0 \\ \gamma^r_1 \\ \gamma^r_2 \end{bmatrix} = 
\begin{bmatrix} 1+\alpha_1^2+\alpha_2^2 & 2 & 2(1+\alpha_1) \\  \alpha_1(\alpha_2-1) & -1 & -(1+\alpha_1 -\alpha_2) \\ -\alpha_2 & 0 & -\alpha_2 \end{bmatrix}
\begin{bmatrix} \sigma^2_\eta \\ \sigma^2_e \\ \sigma_{\eta e} \end{bmatrix}
\end{align*}

\begin{description}
\item[Autocovariances] {\color{mcxs2}on the LHS are features of data and }
\item[consistent estimates of parameters] $\alpha_1, \alpha_2$ {\color{mcxs2}are functions of data only}
\item[Variances and covariance] {\color{mcxs2}on the RHS are the parameters of the UC model}
\item[The UC model] {\color{mcxs2}is} {\color{purple}identified}
\item[Restriction] $\sigma_{\eta e}=0$ {\color{mcxs2}is dispensable as it over identifies the model}
\end{description}

\end{frame}





{\setbeamercolor{background canvas}{bg=purple}
\begin{frame}

\begin{adjustwidth}{-0.5cm}{0cm}
\vspace{8.3cm}\Large
\textbf{{\color{lightgray}UC model and} {\color{white}Beveridge-Nelson decomposition}}
\end{adjustwidth}

\end{frame}
}






\begin{frame}{Beveridge-Nelson decomposition}

\textbf{Definition.}

\smallskip{\color{mcxs2}The} BN {\color{mcxs2}estimate of trend for a time series integrated of order one,} $y_t\sim I(1)${\color{mcxs2}, is defined to be the limiting forecast as horizon goes to infinity adjusted for the mean rate of growth:}
\begin{align*}
BN_t = \lim_{h\rightarrow\infty}\mathbb{E}_t\left[y_{t+h}-h\mu\right]
\end{align*}

\bigskip\begin{description}
\item[$BN_t$] {\color{mcxs2}is a random walk with the same mean growth rate as the observed series}
\item[$y_t - BN_t$] {\color{mcxs2}the deviation from trend is a stationary process}
\end{description}

\end{frame}




\begin{frame}{Forecasting with UC model}

\textbf{One period ahead.}
\begin{align*}
\mathbb{E}_t[y_{t+1}] &= \mathbb{E}_t[\tau_{t+1} + \epsilon_{t+1}]\\
&= \mathbb{E}_t[\mu + \tau_{t} + \eta_{t+1} + \alpha_p(L)^{-1}e_{t+1}]\\
&= \mu + \tau_{t} + \mathbb{E}_t[\alpha_p(L)^{-1}e_{t+1}]
\end{align*}

\textbf{Two periods ahead.}
\begin{align*}
\mathbb{E}_t[y_{t+2}] &= \mathbb{E}_t[\tau_{t+2} + \epsilon_{t+2}]\\
&= \mathbb{E}_t[2\mu + \tau_{t} + \eta_{t+2}+ \eta_{t+1} + \alpha_p(L)^{-1}e_{t+2}]\\
&= 2\mu + \tau_{t} + \mathbb{E}_t[\alpha_p(L)^{-1}e_{t+2}]
\end{align*}

\end{frame}



\begin{frame}{Forecasting with UC model}

\textbf{$h$ periods ahead.}
\begin{align*}
\mathbb{E}_t[y_{t+h}] &= \mathbb{E}_t[\tau_{t+h} + \epsilon_{t+h}]\\
&= \mathbb{E}_t[h\mu + \tau_{t} + \eta_{t+h}+\dots+ \eta_{t+1} + \alpha_p(L)^{-1}e_{t+h}]\\
&= h\mu + \tau_{t} + \mathbb{E}_t[\alpha_p(L)^{-1}e_{t+h}]\end{align*}

\end{frame}


\begin{frame}{UC model and Beveridge-Nelson decomposition}

\begin{align*}
\lim_{h\rightarrow\infty}\mathbb{E}_t[y_{t+h} - h\mu] &= \tau_{t} 
\end{align*}

\bigskip\begin{description}
\item[$\tau_t$] {\color{mcxs2}is interpreted as a trend component in the BN sense} 

\bigskip\item[At the limit] $\lim_{h\rightarrow\infty}\mathbb{E}_t[\alpha_p(L)^{-1}e_{t+h}]=0$ {\color{mcxs2}as the stationarity condition holds and} $h\gg p$
\end{description}

\end{frame}

{\setbeamercolor{background canvas}{bg=purple}
\begin{frame}{Unobserved Component models}

\begin{description}
\item[decompose] {\color{white}the original time series into} 
	\begin{description}\small
	\item[unit-root non-stationary trend] {\color{white}that is linked to  the} $BN_t$ {\color{white}component}
	\item[unit-root stationary deviation from trend] {\color{white}that may exhibit cyclical patterns of persistence (if} $p\geq2${\color{white})}
	\end{description}

\bigskip\item[belong to a family] {\color{white}of state-space models} 

\bigskip\item[Its ARIMA representation] {\color{white}facilitates the interpretation and understanding of properties} 

\bigskip\item[are applied] {\color{white}in particular to modeling trend inflation and\\ to estimation of output gap} 
\end{description}

\end{frame}}


\end{document} 